\section{Obstacles to Reproducibility in Computational Sciences}
\label{s:obstacles}

There are various obstacles to reproducibility, and many of these were discussed by speakers and participants during the week. 
Although not intended to be an exhaustive list,  amongst the obstacles and challenges that we identified are:
technical (long-term persistence of experiments; amount of efforts required from researchers to create reproducible experiments, 
which varies very much for different areas), 
legal and ethical (licensing issues; data protection issues, e.g. with researcher's personal data drifting into the virtual machine 
while it is in use), 
community building (why to make your research reproducible and how to to motivate your colleagues/collaborators/supervisors/etc 
to think the same). 

Other problems that came out from the discussions include:
performance measurement; 
possibly non-trivial amount of knowledge that is still needed to rerun an experiment;
ease of reproduction vs independent reproducibility; 
automation tradeoff (recomputation tends to need automated systems);
ability to get ethical consent (did they ask to release the data in a form we can use?); 
not knowing what data was decided not to 
publish in the paper.



