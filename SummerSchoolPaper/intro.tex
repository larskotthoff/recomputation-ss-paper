\section{Introduction}
\label{s:intro}

This paper has been written during the week of the Summer School
on Experimental Methodology in Computational Science Research
\footnote{St Andrews, Scotland, August 4-8, 2014,
\url{https://blogs.cs.st-andrews.ac.uk/emcsr2014/}}. The 
school was comprising of lectures, tutorials and a hackday,
all united by the topic of reproducible research in computational
sciences. 

The programme included the following lectures and tutorials:
\begin{itemize}
   \item Steve Crouch (Software Sustainability Institute): Developing Sustainable Research Software -- why is it important and what resources are out there?
   \item Ian Gent (St Andrews): Reproducibility in Computer Science
   \item Tristan Henderson (St Andrews): Data Sharing: the good, the bad and the ugly
    \item Neil Chue Hong (Software Sustainability Institute), Better Software, Better Research: Why reproducibility is important for your research
    \item Darren Kidney (St Andrews): Data Analysis using R
    \item Alexander Konovalov (St Andrews): Using Microsoft Azure (Tutorial)
    \item Lars Kotthoff (University College Cork): Recomputation
    \item Miguel Nacenta (St Andrews): Experimental Methodology in HCI Research
    \item Burkhard Schafer (Edinburgh): Legal Issues in Computer Science
    \item Kenji Takeda (Microsoft Research Cambridge), Reproducible Research and the Cloud
\end{itemize}

For the hackday (actually split into several sessions), attendees 
were asked to `bring a paper', i.e. to bring the suggestion of a 
computational experiment to be reproduced as part of the summer school. 
The experiment should be either from a published paper or, if unpublished, 
have the agreement of all authors concerned for this activity including 
the publication of a replication. 

After forming teams and selecting projects for the hackday, three
topics for the case studies were selected. These will be presented below
in Sections~\ref{s:group1}--~\ref{s:group3}.



