\section{Case study \#4: Recomputability of HCI Studies}
\label{s:group4}

Research in computer science often involves humans, especially if the subject of study is how people and machines interact with each other (e.g., in HCI). This kind of research focuses on phenomena that involve human agents and are therefore not addressable strictly through computation. However, there are also strictly computational issues that are critical for the replicability of HCI research. In this case study we looked at current practice in HCI research through an example published experiment, and aimed at identifying the road blocks and difficulties that present themselves when reproducing statistical analysis of data obtained from human behavior.  

\subsection{Background}
The field of Human-computer Interaction heavily relies on the execution and analysis of empirical studies that involve humans. These results are used to, for example, build new interaction techniques and devices\cite{}, and propose theories of human behavior with computers \cite{}. The reliability of the analysis and conclusions of these studies have been recently questioned. Problems identified include the fact that barely any results are replicated \cite{hornbaek}, that the research is often difficult or impossible to replicate \cite{wilson1}, and that replicated results are difficult to publish and share to the larger community \cite{}. The HCI community is currently trying to address some of these problems through new venues for publication (e.g., ReplyCHI~\cite{replichiVenue}), the creation of new tools~\cite{mackay}, and efforts to change the research culture and incentives (ACM CHI, arguably the most important conference in the field, introduced in 2013 a replication award or distinction for papers that address replicability).

Although the problem of replicability of experiments with humans is difficult and will likely require significant efforts from the community, the recomputability of these results and the associated statistical analysis has received relatively little attention, even though it is probably one of the most significant sources of inaccuracy and incorrect data in the field. Recomputability in HCI experiments refers mostly to the ability of others (not authors) to replicate the statistical analysis and statistical conclusions of a paper utilizing the same recorded data from the experiment. 

Quantitative experiments in HCI analyze data that is obtained from humans to draw conclusions that are relevant for the understanding or development of interfaces. Although the experiments are necessarily affected by the inherent variability introduced by humans, the analyses should not. Ideally, every researcher in the area, and more specifically, every reviewer of a paper containing statistical analysis of quantitative human data should be able to reproduce the analysis. The ability of reviewers to determine if a statistical analysis and interpretation of the data in a paper are correct is currently limited to checking that the reported degrees of freedom in an ANOVA (or a similar inferential statistic procedure) are consistent with the design of the experiment, and that the intermediate statistic figures (e.g., F values, DOF, p-values) are consistent with the statistic analysis. This is obviously not sufficient to detect even relatively simple errors during analysis that could mean the difference between radically opposite interpretations of the data. Examples that have been encountered by some of the authors of this paper include: reading statistics and degrees of freedom from an incorrect column, reading statistics and degrees of freedom from an incorrect table, and performing within-subject analysis on between-subjects data. Several of these errors are virtually impossible to detect if the only provided information are the statistical figures typically found in papers. The problem is further magnified if the analysis is not standard. For example, if a new computational measure is created from the data, it might be impossible to reproduce without having the exact code, and if the analysis applies a machine learning approach there might be large numbers of parameters to adjust and many differently implemented variants of the same analysis (different analysis frameworks might have implementations of the same analysis that might lead to different results).

In order to prevent those errors and the significant loss of credibility of the data that they cause, authors should enable the recomputation of statistical and machine learning analysis on the data of any experiment to the extent allowed by other ethics and privacy issues (See section XXXX). This requires that: a) authors make the data available, b) authors provide suitable meta-data that describes the semantics and structure of the data, c) authors provide instruction on how to reproduce the analysis. In the spirit of openness and scientific integrity, one of the authors (M.A.N.) has been striving to provide the data and the analyses for his own empirical research in HCI. Specifically, a recent research project on the memorability of gestures \cite{Nacenta} was developed from scratch with the idea of sharing the 



word vs latex

 interpretation of the data 

 that are often difficult or impossible to replicate, 

Several voices have raised concerns that th data being gathered and the conclusions of many studies might be erroneous, mostly

However, it has been noted that the standards of replicability of published studies is not comparable to other related fields \cite{hornbaek,wilson1}. 

Replicability of experimental results in human-computer interaction experiments


- Replicability in HCI picking up (replichi, 1,2,3,4 kasper, Mackay, prize) 
- Efforts to reproduce experiments
- Little attention to the replication of statistical findings, important parts
- This is the easy part, and the one that might provide for most benefits
- Currently, research process only allows for a cursory examination of DOF
	- typical failures: wrong analysis (SPSS)

\subsection{Experience Report: Recomputing a Memorability Experiment}
- One of the authors: support for replication, 2 papers
- Big questions as to whether even if the data is made available, the process is acceptable

- Authors (except main author of the paper), generally a group of experts, set out to reproduce the main selected results in a 1.5h session
- Presentation on the topic, to allow them to understand
- 

\subsection{Results}
- Layers of knowledge needed pretty broad
- Access to software problematic (SPSS)
- Specific numbers difficult to find
- Consistency between 

\subsection{Discussion}

\subsubsection{Challenges}

\subsubsection{Recommendations}




Human subjects research involves the collection of data through interaction
with individuals, or through collection of personally identifiable
information. Such research poses specific barriers to reproducibility. In
fields such as human-computer interaction (HCI), there are few replications of
previous work, attributable to a culture which does not reward
reproducibility, difficulty in replicating interaction techniques when
materials are not shared, and an emphasis on formative work which proposes new
techniques over summative work~\cite{hornbaek:replications}.

In this section, we look at one specific challenge to reproducibility in HCI
research: capturing and disseminating the ethical requirements of an
experiment, such that others may better replicate the procedures of a
study. We evaluate the ethics requirements procedures of ten universities to
determine a minimum specification for reporting ethical considerations, and
discuss the state of the art in standardising ethical requirements by
professional bodies and research councils.

\subsection{Background}
Human subjects research often involves the collection
of sensitive identifiable data about participants. To ensure participants are
not placed at undue risk by the conduct of an experiment, an increasingly
rigorous process of oversight of the ethical conduct of research institutions
has emerged in recent decades in many countries, particularly the US, where
institutional review boards (IRB) have been charged with reviewing all
clinical and human subjects research in line with federal regulations, with
significant penalties for institutions if ethics violations occur. Each
institution, however, has freedom to implement IRB processes as they see fit
so long as such regulations are upheld. This leads to great inconsistencies
between the expectations of institutions, and the processes researchers must
engage with in order for research protocols to be approved.

Internationally, the situation is even more variable. Many countries may not
impose any such requirements on institutions, while in the UK, research
councils mandate ethics are considered in order to receive funding, however
the conduct of individual ethics committees is not regulated.

Such variety in the conduct of ethical approval between institutions a
represents a significant barrier to reproducibility in HCI research.
If a researcher wishes to replicate a HCI experiment which uses human
subjects data, they will usually need to seek ethical approval from
their own institution. In our recent work, a survey of 505 papers
using online social network (OSN) data found that only 2\% of papers
disclose any of the ethical considerations of their
work~\cite{hutton:reproducibility}.

% cite osn survey!

As researchers do not routinely disclose the protocol which
received IRB approval, attempted replications may miss crucial details
necessary to conduct the experiment, and to provide to IRB when seeking
ethical approval, which can reduce the correctness of a replication.

With IRBs and ethics boards operating largely independently with little policy
coordination, there is no standardisation of ethics procedures.

Indeed, reproducibility has only recently been considered an important
ambition for HCI researchers, with nascent efforts including RepliCHI, which
has operated as a CHI workshop since 2013. Despite such efforts however, the
wider community has not considered these ethical challenges in detail.

Next, we assess the state of the art in ethics procedures to determine what
commonalities exist between institutional requirements. From this, we aim to
derive a minimum ``ethical specification'', encoding fundamental
methodological details to help researchers replicate procedures and ethical
details, and to make it easier to replicate applications to other IRBs, with
the ambition of such specifications being routinely attached to HCI
experiments.


\subsection{Comparison of Ethical Requirements across Universities}

To understand the state of ethics procedures between institutions, we
collected ethics applications forms for ten public universities located in the
UK, EU, and USA, and Asia. A range of locations were chosen to capture a range of
cultural and regulatory expectations, which we expect will manifest in
different procedures. All forms collected were derived from publicly
accessible sources, except for one supplied by the authors. This is in
itself a significant barrier to reproducibility. Without making procedures publicly 
available, there can be no external scrutiny about the appropriateness of some institution's
procedures, and makes it more difficult to derive standards.

For each form, two researchers independently coded each field, accounting for
differences in wording between forms so long as each attribute asked for the
same atomic information. Where one form requests expanded information
pertaining to a previously identified attribute, this was added as an
additional attribute. After independently coding the forms, the two
researchers discussed any discrepancies to arrive at a set of 145 unique
attributes, encompassing generic details, such as contact details of co-
investigators, methodological details, and institution-specific requirements,
often for insurance and liability purposes. Of these fields, only two were
common to all ten ethics forms - the name of the principal investigator, and
whether informed consent was sought. This intersection was significantly
smaller than anticipated, and clearly does not constitute a useful minimum
ethical specification. It does however reveal two interesting properties. It
confirms our intuition that ethical procedures vary greatly between
institutions, while also identifying perhaps the single most important
objective of the ethics process: to ensure participants have given informed
consent to participate in an experiment.

\begin{figure}
\begin{minipage}{\linewidth}
\includegraphics{group1-001}
\caption{\label{p:ethicsFreq}Histogram showing frequency distribution of fields in ethics applications}
\end{minipage}
\end{figure}

Figure~\ref{p:ethicsFreq} shows the distribution of attributes across the ten
forms we examined. As shown in the long tail, 47.6\% of attributes only appear
in one form. Of these, 42\% are ``sub-attributes'', that is, requesting
additional information requested on another form. For example, while 60\% of
forms ask whether participants receive financial compensation, only one asks
whether co-investigators are compensated. We are most interested in the unique
``high-level attributes'' which emerge, as it is important to discern between
questions which capture institution-specific requirements, or may constitute
important issues which other IRBs ought to consider. We find instances of both
in our results. For example, while UCL are the only institution to ask whether
their own students are participants in the research (we assume for liability
reasons), surprisingly they are the only university to ask outright whether
health and safety precautions have been considered. Interestingly, only Aga Khan
University in Pakistan asks whether the study is a replication of a previous experiment.

\subsection{Proposed ethics specification}
Given the small intersection of
attributes in our study, we isolate the twenty most common attributes -- those
which occurred in six or more of the ten forms we examined. We combine any
semantically similar fields to produce the following set of 15 attributes,
presented in descending order of frequency.

\begin{itemize}
	\item Was consent sought?
	\item Was deception involved?
	\item Project title
	\item Study duration
	\item Are there risks to participants?
	\item Justify use of vulnerable participants
	\item PI contact details
	\item Funding body information
	\item Is likely to induce participant stress?
	\item Summarise research proposal/experimental methods
	\item Are supplementary documents attached? (consent forms, briefing info etc.)
	\item Are participants financially compensated?
	\item Is study clinical?
	\item Supervisor name
	\item Describe ethical issues 
\end{itemize}

This set of attributes covers a range of fundamental methodological details,
many of which can be encoded in a consistent fashion and attached as metadata
to support replications. In further work, we will demonstrate whether this set
of attributes is sufficient to capture key methodological details, which we
cannot assert from this strictly frequency-based exercise.

\subsubsection{Limitations}
This analysis is not intended as a rigorous survey
of ethics procedures internationally. The selection of institutions is
inherently biased, as we are only able to extract forms which are publicly
accessible, a barrier to reproducibility, and we have a particular emphasis on
the UK and US in this study. The intent of this exercise is not to make
statistical inferences about the state of the art in ethics review, but rather
to motivate the minimum set of attributes we recommend researchers disclose
when sharing their experimental methodologies.



