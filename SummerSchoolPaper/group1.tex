\title{Case study \#1: Ethical requirements for recomputation}
\author{
	Benjamin M. Gorman \\
               School of Computing,
        University of Dundee\\
        Dundee DD1 4HN, \underline{Scotland}\\
	b.gorman@dundee.ac.uk\\
	\and\\
	Luke Hutton \\
		School of Computer Science,
		University of St Andrews\\
		St Andrews KY16 9SX, \underline{Scotland}\\
	lh49@st-andrews.ac.uk	
	\and\\
        Karen Petrie \\
               School of Computing,
        	      University of Dundee\\
        Dundee DD1 4HN, \underline{Scotland}\\
        kpetrie@dundee.ac.uk 
       %add yourself here! 
}
\date{\today}

\documentclass[12pt]{article}

\begin{document}
\maketitle

\section{Introduction}
The goal of recomputation\footnote{http://www.recomputation.org/} is to follow the recompilation manifesto\footnote{http://www.recomputation.org/blog/2013/04/12/the-recomputation-manifesto/} and make computational experiments recomputable by providing tools and a repository to store experiments in. The mission is stated as:
\begin{quote}
``If we can compute your experiment now, anyone can recompute it 20 years from now''
\end{quote}

One of the considerations perhaps under considered in regards to recomputation to date is the ethical implications. 
\section{Background}

\section{Comparison of Ethical Requirements across Universities}

\section{Comparison of Ethical Requirements across Publishers}

\section{Framework for Ethics to accompany data and experiments}

\section{Conclusion}

\end{document}
