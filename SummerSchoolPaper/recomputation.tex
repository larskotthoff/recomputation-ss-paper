\section{The state of the art}
\label{s:recomputation}

Reproducibility, replicability and the like have been discussed for
many years~\cite{lubin:replicability} and are acknowledged as a
fundamental part of the scientific method. More recently, interest has
arisen in reproducibility specifically in the computational sciences, with
some considering the nature of the computational sciences to make the
issues distinct from other sciences~\cite{donoho:reproducible}.
Despite its importance, various studies have shown that many, if not
most, research papers are not
reproducible~\cite{bonnet:repeatability,hornbaek:replications,ioannidis:repeatability}.
This has led to calls for independent boards to replicate and certify
research experiments~\cite{baker:verify}. At the same time, efforts
are ongoing to improve methodologies into determining whether an
experiment has been reproduced accurately~\cite{johnson:evidence}.

Terminology is important, but unfortunately many terms are used
interchangeably in this area.  Peng~\cite{peng:reproducible} describes
a spectrum of reproducibility from a paper that is not reproducible,
to one that allows ``full replication''. But
Drummond~\cite{drummond:replicability} distinguishes between
replicability and reproducibility, by saying that the former is the
exact repeating of an experiment as presented, whereas reproducibility
is broader and allows one to build on an experiment and further
science. Gent introduces another term, \emph{``recomputation''}, to describe
the replication of computational experiments (The Recomputation
Manifesto~\cite{gent:recomputation}).
In this paper we consider challenges and ways in which we can enable
the widespread recomputation of experiments as described in the
Recomputation Manifesto.

There are many aspects to recomputation, all of which are being actively
studied. Stodden and Miguez propose a set of best practices for reproducible and
extensible research, including licensing and sharing of data, workflow tracking,
making code and method available, and citing data and
software~\cite{stodden:practices}. Companies such as
FigShare~\cite{figshare:citable} and GitHub~\cite{github:citable} are helping by
making it easier for researchers to share code and data and also cite them
through the use of DOIs (Digital Object Identifier). Several initiatives have
arisen to make researchers more aware of tools, and to further the development
of tools, to make their research more reproducible, e.g., Mozilla Science
Labs~\cite{mozillasciencelabs},
Open Science Framework~\cite{openscienceframework},
RunMyCode~\cite{runmycode}
and Software Carpentry~\cite{softwarecarpentry}.
Davison describes how
to make it easier to capture workflow and experimental context, including using
a ``consistent, repeatable computing environment'' (the recomputation.org
project and others~\cite{howe:reproducible} aim to make this particular aspect
easier through the use of VMs), version control and clearly
separated experimental parameters~\cite{davison:reproducibility}. Mesirov
describes one particular workflow to make it easy to track and package genomic
data~\cite{mesirov:accessible}. Our paper attempts to follow these best
practices, uses version control and is available to recompute on a VM.

%Each of the case studies presented in this paper builds on the current
%state of the art. While others have looked at code, method and data
%sharing, we highlight one particularly overlooked part of method that
%also needs to be shared: the ethics process. As virtual machines are
%increasingly used for packaging and making experiments recomputatable,
%we highlight the issues that arise when such VMs are used for parallel and
%distributed experiments. As the computational sciences encompass more
%disciplines than just computer science, we focus on whether any issues
%arise when recomputating non-CS experiments. And finally, we study one
%particular part of computer science that straddles other disiplines
%such as psychology, to see if there are any further challenges to
%reproducibility here.
%OLD TEXT follows:
%In this section we are going to describe the current state of reproducible computational science, and
%in particular the Recomputation.org project, which is aimed at implementing
%the infrastructure for recomputable experiments following the statements of the Recomputation
%Manifesto \cite{gent:recomputation}. We will give an overview of other activities in this
%direction, such as, for example, the Mozilla Science Lab, GitHub and Figshare joint project to fix code
%citations by assigning a DOI to the particular revision, and training initiatives such as Software
%Carpentry to increase the awareness of researchers about tools that they can use to make their
%research more reproducible -- in other words, promote ``Reproduciliteracy'' (a word coined by 
%Ian Gent at one of the group discussions at the Collaborations Workshop 2014).
%
%\emph{This section is in a preliminary state}
%
