\section{The state of the art}
\label{s:recomputation}

Reproducibility in computational science has been of interest for
several years now, with some considering the nature of computational
science to make the issues distinct from other
sciences~\cite{donoho:reproducible}. While reproducibility is
acknowledged as a fundamental part of the scientific method, various
studies have shown that many, if not most, research papers are not
reproducible~\cite{bonnet:repeatability,hornbaek:replications,ioannidis:repeatability}.

Terminology is important, but unfortunately many terms are used
interchangeably in this area.  Peng~\cite{peng:reproducible} describes
a spectrum of reproducibility from a paper that is not reproducible,
to one that allows ``full replication''. But
Drummond~\cite{drummond:replicability} distinguishes between
replicability and reproducibility by saying that the former is the
exact repeating of an experiment as presented, whereas reproducibility
is broader and allows one to build on an experiment and further
science. Gent introduces another term, ``recomputation'', to describe
the replication of computational experiments~\cite{gent:recomputation}.
In this paper we consider challenges and ways in which we can enable
the widespread recomputation of experiments as described in the
Recomputation Manifesto.

There are many aspects to recomputation, all of which are being
actively studied. Stodden and Miguez propose a set of best practices,
including licensing and sharing of data, workflow tracking, making
code and method available, and citing data and
software~\cite{stodden:practices}. Davison describes how to make it
easier to capture workflow and experimental context, including using a
``consistent, repeatable computing environment'' (the
recomputation.org project and others~\cite{howe:reproducible} aim to
make this particular aspect easier through the use of virtual
machines), version control and clearly separated experimental
parameters~\cite{davison:reproducibility}. Our paper attempts to
follow these best practices, having used version control and is
available to recomputate on a VM.  Mesirov describes one particular
workflow to make it easy to track and package genomic
data~\cite{mesirov:accessible}.

OLD TEXT follows:
In this section we are going to describe the current state of reproducible computational science, and
in particular the Recomputation.org project, which is aimed at implementing
the infrastructure for recomputable experiments following the statements of the Recomputation
Manifesto \cite{gent:recomputation}. We will give an overview of other activities in this
direction, such as, for example, the Mozilla Science Lab, GitHub and Figshare joint project to fix code
citations by assigning a DOI to the particular revision, and training initiatives such as Software
Carpentry to increase the awareness of researchers about tools that they can use to make their
research more reproducible -- in other words, promote ``Reproduciliteracy'' (a word coined by 
Ian Gent at one of the group discussions at the Collaborations Workshop 2014).

\emph{This section is in a preliminary state}

\cite{johnson:evidence}
