\section{Discussion and Conclusions}
\label{s:discussion}

The experiences we have reported on in this paper cannot be taken as full scale studies from which conclusions can definitely be drawn, but aim to provide evidence and some systematic reflection of the current state of scientific reproducibility as seen from the researchers viewpoint.  For example, the 
selection of ethics forms we obtained was certainly subject to selection bias: they were typically the ethics forms that it is easy to obtain.  So we cannot state hypotheses and say our analyses definitively show the hypothesis is supported.   Having given that caveat, we feel that our investigations have uncovered a number of interesting points.  The remarkable diversity of ethics forms enforced around the world raises serious issues for reproducibility.  It also suggests an unnecessary cause of inertia in reducing the speed with which experiments can be started up.  Our parallel experiments suggest that the difficulties of reproducibility are 
multiplied by the extra complications inherent in this domain.  Our
recomputation of non computer science experiments was encouraging in
that it was achieved in a relatively short amount of time.
Nevertheless important issues arose, such as the impossibility of
distributing the resulting VM because of legal issues. 
Such issues need attention and easing as much as possible.  In many cases literally nobody knows what the answers to some of the legal questions are: this is not because they are necessary complex but because they have not been specifically assessed by lawyers, or ultimately tested in court.  This means that we have a situation where we suspect that how we are reproducing work is legal, but do not know it is: a position that might be comfortable in quantum mechanics but less so to the more classically binary computer scientist. 

\textbf{Say something about Miguel's section?}


