\section{The EMCSR summer school}
\label{s:school}

The summer school comprised two activities: a set of speakers
describing ongoing developments and challenges in reproducible
research, and a set of projects chosen by participants to address some
of these challenges.

Ian Gent from St Andrews and Lars Kotthoff from Cork gave
presentations outlining the current state of the recomputation
project, covering such issues as what it means to ``recompute'' a
computational experiment, and the requirements for storing and
reproducing an experiment using virtual machines.

Kenji Takeda from Microsoft Research delved further into the use of
virtualisation, by describing the impact of cloud computing on
computational research.\footnote{e.g., the various projects at
\url{http://azure4research.com/}} As more and more research projects
involve the use of virtual machines running on cloud infrastructure,
it may be possible to leverage this for creating
reproducible experiments by sharing these virtual machines. After that,
Alexander Konovalov from St Andrews gave a tutorial on creating
virtual machines in Microsoft Azure (6-month passes to access the
latter were provided free of charge to all participants thanks to Microsoft Research's
generosity).

Two members of the Software Sustainability
Institute\footnote{\url{http://www.software.ac.uk/}}, Neil Chue Hong
and Steve Crouch, discussed the role of sustainable research software
in reproducible computational experiments. While Neil focused on the
need for reproducibility, Steve discussed how to make research
software sustainable, by using best practices for programming, and
indeed thinking about what ``good'' code means.

Darren Kidney from St Andrews covered more practical aspects of reproducible
research and gave a tutorial on R and SWeave to create
recomputable papers. Indeed this paper is written using SWeave.

In addition to sharing code and the systems used to run an experiment,
much reproducible research requires the sharing of research data,
which in itself involves many challenges. Tristan Henderson from St
Andrews discussed his experiences in running the CRAWDAD wireless
network data archive\footnote{\url{http://crawdad.org/}}, and the
difficulties in documenting data, convincing researchers to dedicate
the time to share their data, citing data correctly and the privacy
issues surrounding the sharing of sensitive data.

Sharing code, data or systems often involves legal issues: code
may have particular licenses that make sharing hard or easy, data may
be subject to data protection legislation, or indeed the law might
suggest best practices that could make recomputable research easier to
achieve.  Burkhard Schafer from the Edinburgh School of Law gave a set
of presentations covering these and many other aspects of UK law.

Not all recomputable computational experiments involve only
computations. Miguel Nacenta from the St Andrews Human-Computer
Interaction research group discussed the issues around
reproducibility in HCI experiments that involve human participants,
and led a session in which summer school attendees attempted to
reproduce the analysis from one of his own HCI research papers.
