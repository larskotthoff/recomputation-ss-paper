\section{Open, Executable, Reproducible}

This paper is intended to be \emph{open}, 
\emph{executable},
and \emph{reproducible}.  
We discuss here what we mean by these words, how we tried to achieve them, and to what extent we succeeded.

By ``open'', we mean that the paper was openly developed and written.  The paper was first sketched out in the week of the summer school, and then 
developed further over the following three weeks.  The authors collaborated via GitHub: this is not unusual but distributed source code control was  particularly important with so may authors working in parallel over a short period of time.
Because we used a public GitHub repository~\cite{summerschoolpaper}, 
the
development of the paper can be tracked throughout through the commit history on
GitHub.  
Anybody can download not only the final paper, but its source, and code and other materials collected during its development.  
An interesting aside, relevant to cases where statements of contribution are required, is that anybody can see what each author committed on the paper or supporting materials throughout its life.  It must be borne in mind, however, that a commit by one person may represent the work of several authors working together offline.  

By ``executable'', we mean that the paper that can be reconstructed from source 
materials, and that data can be reanalysed as it changes and new versions of the paper produced, and possibly executable code rerun.   The name has been used, for example, by the ``Executable Papers Grand Challenge''~\cite{executablepaper}.
This has numerous advantages because as we add data, the paper does not need to be rewritten.\footnote{The name ``reproducible paper'' is sometimes used for this, but can lead to confusion because a paper can be executable in the sense of being able to produce new figures with 
changed data, but not reproducible if that data can not be reconstructed ab initio.}
To make our paper executable, we used Sweave~\cite{lmucs-papers:Leisch:2002}, a package that integrates the statistics system R and \LaTeX. 
To ease the workflow and to make execution of the paper easy, we wrote a Makefile for the generation of the paper PDF, although there are still some issues which need manual intervention such as installation of the necessary R packages. 

By ``reproducible'', we mean that it is our intention that other scientists (or
ourselves at later dates) will be able to reproduce our work to assess if
statements we make are correct and if conclusions are valid.  
As mentioned earlier, 
To enable this we
have attempted to collate materials necessary for each study, and make them
available to future researchers.  
In some cases this has also been done in git,
with materials such as ethics forms and experimental results put into the repository.  
In other cases we have constructed VMs to recompute experiments.  
We also built a machine which not only contains a clone of the GitHub repository, but all the necessary software and packages, such as R and \LaTeX, to build the paper.
As well as reproducibility, this helped us during the writing of the paper.  At times some authors would be unable to build the whole paper, perhaps because of slight differences in version numbers of R or its packages from other authors.  The availability of the VM in which the paper could build was invaluable, since if it built there, we were safe.  This also highlights the value of the paper VM, since if not all authors could build the paper during its preparation because of package inconsistencies, it is very likely that future workers would not be able to build it from the GitHub repository without some work.
Even a small VM 
might be half a gigabyte, and such large files can be problematic for git.
These are therefore stored elsewhere.  
Our approach can be seen as similar to that of Brown~\cite{brown} for one of his papers with colleagues~\cite{Brown2012}.


To what extent have we succeeded in our efforts?
Our success is mixed.  We cannot be completely open because
some of the data and/or programs used in various parts of our paper do not allow us to share them.  This is particularly true of the third case study, and as a result we cannot distribute the VM embodying those experiments. Also our paper is not fully executable since 
many computations involved in constructing the data must be run by hand.  Also some of the tables in this paper are static, with data entered manually rather than reanalysed by Sweave. This contrasts negatively with Brown's executable paper~\cite{brown}.  
In terms of reproducibility, we feel we have been mostly successful.  
Provision of materials via GitHub will be helpful for those who wish to adapt or reproduce our work.
We have provided VMs where possible and appropriate, and can distribute two of them: this should enable the recomputation of our paper.  This can be performed either locally, or using VMs deployed in Microsoft Azure requiring no software installation on a users machine. 
While we might pat ourselves on the back for this, we feel it is better to be cautious. We ourselves have discussed potential issues, such as ethical and legal, and unforeseen technical issues might prevent reproducibility. The real test of reproducibility must be time.
It is interesting to speculate: for example, if we run a second summer school in 2015, we might ask participants to reproduce this paper and see how we did this year!

\textcolor{red}{Put in links somewhere to VMs: recomputation.org and Azure VM Depot.\footnote{http://vmdepot.msopentech.com/Vhd/Show?vhdId=44582} Should we try and get a GitHub DOI?}


